\documentclass[convert]{standalone}

\usepackage{csvsimple} % read and display csv files as tables
\usepackage{booktabs} % prettier tables
\usepackage{threeparttable} % easy addition of footnotes below tables
\usepackage{array} % using command \arraybackslash to restore the definition of \\ after \centering

\newcommand{\cit}{\centering\textit}
% Reducing the space between columns
% \setlength{\tabcolsep}{4pt}

% END OF PREAMBLE

\begin{document}

% Table 2
	\begin{threeparttable}\small

		\csvreader[tabular = l*{3}{*{2}{>{\raggedleft}p{0.07\textwidth}@{\hskip 2pt}>{\raggedright}p{0.07\textwidth}}c}, 
		head to column names,
		table head = \toprule & 
		                        \multicolumn{4}{c}{REF\tnote{a} (\%)} & & %
					\multicolumn{4}{c}{DEL\tnote{b} (\%)} & & %
					\multicolumn{4}{c}{INS\tnote{c} (\%)} \\ %
					\cmidrule{2-5} \cmidrule{7-10} \cmidrule{12-15}%
		TE type & %
		\multicolumn{2}{c}{\cit N\tnote{d}} & \multicolumn{2}{c}{Mb\tnote{e}} &  & %
		\multicolumn{2}{c}{\cit N} & \multicolumn{2}{c}{kb\tnote{f}} & & %
		\multicolumn{2}{c}{\cit N} & \multicolumn{2}{c}{kb} \arraybackslash\\\midrule,
		table foot = \bottomrule]%
			{table_2.csv}%
			{}%
			{\type & \ref & (\refprop) & \refmb & (\refmbprop) & & \DEL & (\DELprop) & \DELkb & (\DELkbprop) & & \INS & (\INSprop) & \INSkb & (\INSkbprop) \arraybackslash}

			   \begin{tablenotes}\footnotesize
			   \item[a] REF: transposable elements $\geq$ 100 bp in the reference genome
			   \item[b] DEL: deletions relative to the reference that are annotated as TEs
			   \item[c] INS: insertions relative to the reference that are annotated as TEs
			   \item[d] \textit N: Number of reference elements, deletions or insertions matching given TE type %(\% of matching SVs corresponding to a given TE type)
			   \item[e] Mb: Total length of reference elements of a given type, in Mb
			   \item[f] kb: Total length of polymorphic elements matching given TE type, in kb %(\% of polymorphic bases corresponding to a given TE type)
			   \end{tablenotes}

	\end{threeparttable}

\end{document}

