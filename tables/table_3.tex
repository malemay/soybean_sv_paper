\documentclass[convert]{standalone}

\usepackage{csvsimple} % read and display csv files as tables
\usepackage{booktabs} % prettier tables
\usepackage{threeparttable} % easy addition of footnotes below tables
\usepackage{calc} % makes \widthof available to set the width of the last column
\usepackage{array} % using command \arraybackslash to restore the definition of \\ after \centering

% Reducing the space between columns
% \setlength{\tabcolsep}{4pt}

% END OF PREAMBLE

\begin{document}

% Table 3

	\begin{threeparttable}\small

		\csvreader[head to column names,
		tabular = lm{\widthof{\% in reference}}llcm{\widthof{\% in reference}}ll,
		table head = \toprule & \multicolumn{3}{c}{Whole genome} & & \multicolumn{3}{c}{Non-repeated genome}\\ \cmidrule{2-4} \cmidrule{6-8}%
		Feature & \% in reference & DEL (\%) & INS (\%) & & \% in reference & DEL (\%) & INS (\%)\\\midrule,
		table foot = \bottomrule]%
		{table_3.csv}%
		{}%
		{\feature & \centering \gprop & \DEL~(\DELprop) & \INS~(\INSprop) & &%
		\centering \nrgprop & \nrDEL~(\nrDELprop) & \nrINS~(\nrINSprop)\arraybackslash}

			   \begin{tablenotes}\footnotesize
			   \item{ } DEL: deletions, INS: insertions
			   \item{ } cds: coding sequence
			   \item{ } upstream5kb: within 5 kb upstream of a gene
			   \end{tablenotes}

	\end{threeparttable}

\end{document}

